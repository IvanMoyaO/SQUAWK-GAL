\section*{Ecuaciones}
El código está disponible en GitHub (\textcolor{blue}{\href{https://github.com/IvanMoyaO/SQUAWK-GAL}{https://github.com/IvanMoyaO/SQUAWK-GAL}}) y en las siguientes páginas. 

La estructura del código es la siguiente:
\begin{itemize}
    \item \textbf{Cabecera y declaración de entradas y salidas}: en las primeras 36 líneas, se establece la cabecera del programa (con información como la fecha o el dispositivo) y se declaran qué pines se usarán como entradas/salidas y qué variable tendrán asignada. Nótese que la variable los pulsos, $Z$, se asigna al pin 18 en concreto debido al número de términos que tiene.
    \item \textbf{Ecuaciones del contador}: entre las líneas 38 y 58 se realizan las ecuaciones de un contador sencillo, sin truncar y que cuenta todos lo números.
    \item \textbf{Variables intermedias}: entre las líneas 63 y 77 se declaran una serie de variables intermedias que corresponden a los instantes en los que habrá un pulso. Esta parte no es necesaria, sin embargo, consideramos que favorece enormemente la legibilidad.
    \item \textbf{Salida}: en la línea 79 se establece el valor (1 ó 0) de la salida. Solo se activará en determinados instantes si, y solo si, el correspondiente microinterruptor está activo.
\end{itemize}

\begin{cuidado}{Errores que cometimos}
En la primera versión del código, \textcolor{blue}{\href{https://github.com/IvanMoyaO/SQUAWK-GAL/commit/ac02036bc24f1b977cd46d78ace5171c46c0033c}{como puede verse en GitHub}}, los pulsos que generamos no se asemejaban a un código SQUAWK debido a que faltaba $X$, $F_1$ y $F_2$.

\textcolor{blue}{\href{https://github.com/IvanMoyaO/SQUAWK-GAL/commit/d1294f0a59ebac4c123ad7443a96185edbf2dbca}{Una versión posterior}} añadió dichos pulsos, pero seguía siendo incorrecta al no dejar un <<\textit{tiempo muerto}>> tras $F_2$. \textcolor{blue}{\href{https://github.com/IvanMoyaO/SQUAWK-GAL/commit/8cf73270ac348ffd79384d2d9c017e38601c2216}{Los cambios que introdujo definitiva pueden verse en GitHub}}.

\end{cuidado}

El código/ecuaciones se recoge a continuación:

\lstset{inputencoding=utf8/latin1}
\lstinputlisting[
frame=single,
numbers=left,
breaklines=true
]{ecuaciones.pld}

